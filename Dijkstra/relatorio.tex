\documentclass[10pt,a4paper]{article}

\usepackage{arev}
\usepackage{indentfirst}
\usepackage{amsthm,amsfonts,amsmath,amssymb}
\usepackage[brazilian]{babel}
\usepackage[T1]{fontenc}
%\usepackage[latin1]{inputenc}
\usepackage[utf8]{inputenc}
%\usepackage{multicol}
\usepackage{geometry}
\geometry{a4paper,inner=2cm,outer=2cm,top=2cm,bottom=2cm}
\usepackage{setspace}
\usepackage{natbib}
\usepackage{pgf,tikz}
\usepackage{pgfplots}
\usepackage{listings}
%\usepackage{color}
%\usepackage{algpseudocode}
%\usepackage{algorithm}
\usepackage{float}
\usepackage{graphicx}
\usepackage{subfigure}
\usepackage{wrapfig}
\usepackage[linesnumbered, ruled, portuguese]{algorithm2e}
\usepackage{multirow}
%\usepackage{verbatim}
%\usepackage[active,tightpage]{preview}
%\PreviewEnvironment{tikzpicture}
%\setlength\PreviewBorder{5pt}

%\setlength{\oddsidemargin}{-0.8cm}
%\setlength{\textwidth}{17,4cm} \setlength{\textheight}{25cm}
%\setlength{\topmargin}{-2.5cm}

\newcommand{\pr}{\hspace*{0.6cm}}
\newcommand{\vesp}{\vspace*{.3cm}}

\newcommand{\sen}{\mbox{\,sen}}
\newcommand{\cotg}{\mbox{\,cotg\,}}
\newcommand{\tg}{\mbox{\,tg\,}}
\newcommand{\cose}{\mbox{\,cos\,}}
\newcommand{\expo}{\mbox{\,e\,}}
\newcommand{\logg}{\mbox{\,log}}
\newcommand{\Sum}{\displaystyle\sum}
\newcommand{\Prod}{\displaystyle\prod}
\newcommand{\Int}{\displaystyle\int}
\newcommand{\dint}{\, \mathrm{d}}
\newcommand{\Lim}{\displaystyle \lim}
\newcommand{\Frac}{\displaystyle\frac}

\newcommand{\Nc}{N_{cont}}
\newcommand{\Ni}{N_{int}}
\newcommand{\Ne}{N_{estrela}}

\newcommand{\Dparc}[2]{\dfrac{\partial #1}{\partial #2}}
\newcommand{\Dparcn}[3]{\dfrac{\partial^#3 #1}{\partial^#3 #2}}

\newcommand{\R}{\mathbb{R}}
\newcommand{\V}{\mathcal{V}}
\newcommand{\I}{I_u}
\newcommand{\Fi}{\varphi}
\newcommand{\se}{\mbox{ se }}
\newcommand{\norma}[1]{\left|\left| #1 \right|\right|}
\newcommand{\sistema}[1]{ \left\{ #1 \right. }

\newtheorem{exemplo}{\pr \sc Exemplo}[section]%[chapter]
\newtheorem{defi}{\pr \sc Defini\c{c}\~ao}[section]%[chapter]
\newtheorem{obs}{\pr \sc Observa\cao}[section]%[chapter]
\newtheorem{teor}{\pr \sc Teorema}[section]%[chapter]
\newtheorem{lema}{\pr \sc Lema}[section]%[chapter]
\newtheorem{prop}{\pr \sc Proposi\cao}[section]%[chapter]
\newtheorem{exercise}{\pr \sc Exerc\'\i cios}[section]%[chapter]
\newtheorem{alg}{\pr Algoritmo}[section]%[chapter]

\setlength{\columnsep}{1cm}

\setlength{\columnsep}{1cm}


%\author{\textbf{Aluna:} Camila Faria Afonso Lages - 6419748}
%\textbf{Professor:} José Alberto Cuminato\\
%\textit{Instituto de Ciências Matemáticas e de Computação - ICMC}\\ \textit{USP - São Carlos} \\}

\lstset{language=C}

\lstdefinestyle{customc}{
  belowcaptionskip=1\baselineskip,
  breaklines=true,
  xleftmargin=\parindent,
  language=C,
  showstringspaces=false,
    keywordstyle=\bfseries\color{green!40!black},
  commentstyle=\itshape\color{purple!40!black},
  identifierstyle=\color{blue},
  stringstyle=\color{orange},
} 

%basicstyle=\footnotesize\ttfamily,

\lstset{escapechar=@,style=customc}

\begin{document}

\onehalfspace

\thispagestyle{empty}
\hrule \vspace{0.4pc} \hrule
\bigskip
\begin{center}
 {\Large  \textbf{Dijkstra}}
\end{center}


\begin{center}
	\large{ \textbf{Aluno:}  \begin{tabular}{ccc}
							Adams Vietro Codignotto da Silva  & - & 6791943\\ 
					 \end{tabular} 
		\vspace{0.2cm}
		
	\textbf{Professor:} { Alneu} }
		\vspace{0.2cm}
		
	 \end{center}
\bigskip
\hrule \vspace{0.4pc} \hrule

\newpage

\pagenumbering{arabic}

\section{Introdução}

O algoritmo de Djikstra é utilizado para descobrir o menor caminho entre dois vértices, dado um grafo com arestas cujos pesos são positivos. Este programa executa o algoritmo dado um arquivo de entrada contendo a quantidade de arestas e seus respectivos pesos.

\section{Objetivos}

Utilizar o mapa rodoviário dado e interpretá-lo como um grafo. Após isso, utilizando o algoritmo de Djikstra, encontrar a menor distancia de uma cidade a outra, no melhor desempenho possível.

\section{Implementação}

\subsection{Estrutura}
Foram utilizadas duas estruturas: uma lista de vértices (um para cada cidade), e uma matriz. Ambos são declarados estaticamente, porém, modificando o cabeçalho VERTICES no arquivo header, pode-se alterar o tamanho de ambos sem causar danos ao algoritmo.
A lista dos n vértices é composta pela struct grafo, sendo este declarado globalmente e estático:
%CÓDIGO
\begin{lstlisting}
typedef struct
{
    int distancia; //distancia do node ate a origem
    int existe; //se ainda esta no grafo ou nao
    int anterior; //anterior no caminho
} grafo;
\end{lstlisting}

O grafo é montado a partir de um arquivo de entrada (também declarado no cabeçalho \textit{\textbf{INPUT}}), o qual deve conter na primeira linha a quantidade de cidades, e depois o nome das cidades (sem espaços) em cada linha.

A matriz é declarada globalmente como \textit{\textbf{int MPeso[VERTICES][VERTICES]}}, e esta é utilizada para representar o peso entre o vértice $i$ e $j$: ou seja, o campo \textit{\textbf{MPeso[i][j]}} contem o peso da aresta entre os vértices $i$ e $j$. Caso este seja 0, não há uma aresta entre os mesmos. Esta é montada a partir de um arquivo de entrada (declarado no cabeçalho \textit{\textbf{INPUT2}}), que contem na primeira linha a quantidade de arestas (caminhos) existentes, e após isso a cidade de origem, destino e seu peso em cada linha.

\subsection{Algoritmo}

O Djikstra utilizado percorre n vértices vizinhos ao vértice de origem, procurando o menor caminho e o removendo (simplesmente setando a flag \textit{existe} como 0, indicando que este é inexistente). Após isso, verifica se o peso deste é menor que o peso previamente encontrado, se a aresta entre o vértice atual e de origem existe, e se é necessário fazer o relaxamento do mesmo. Isso é repetido ate que o vértice de destino seja encontrado.

\newpage
%CÓDIGO AQUI
\begin{lstlisting}
void dijkstra()
{
    int interacao;
    for(interacao=0; interacao<arestas; interacao++) //enquanto existirem arestas a serem percorridas
    {
        int i;
        int v = acha_menor(); //v eh a menor aresta vizinha a source
        G[v].existe = 0; //Remove a menor aresta de G
        for (i = 0; i < arestas; i++){
            if (G[i].existe)//vertice deve estar no grafo
                if(MPeso[v][i] != 0)//a aresta entre o menor e o atual deve existir
                    if((G[v].distancia + MPeso[v][i] < G[i].distancia || G[i].distancia == 0))
                    {
                        //caso a distancia do menor + distancia entre o atual e o menor
                        //seja menor que a distancia do no sendo visitado ou caso ela nao tenha sido inicializada
                        /*----RELAXAMENTO DE [V][I]----*/
                        G[i].distancia = G[v].distancia + MPeso[v][i]; //vertice sendo visitado recebe peso do menor + distancia entre menor e sendo visitado
                        G[i].anterior = v; //caminho mais proximo eh salvo
                    }
        }
        if(v==dest) {
                return;
        }
    }
}
\end{lstlisting}

O resultado é uma lista invertida do caminho de origem ate o destino (que depois é invertida novamente e o caminho é finalmente impresso). A lista inicia-se no vértice de destino e é percorrida de trás para frente, onde \textit{\textbf{G[i].anterior}} contém o índice do próximo vértice a ser visitado.

\section{Resultados}

O algoritmo se mostrou bastante estável, com complexidade $O(n\log(n))$, onde n é a quantidade de vértices. Os caminhos, sempre que existentes, são garantidos de serem encontrados.

A implementação do algoritmo utilizando uma estrutura de grafos em matriz se mostrou relativamente simples, se comparado com uma estrutura de grafos em listas. A estrutura de matriz acaba utilizando mais espaço de memória, porém a velocidade dos acessos são muito mais rápidas.

\section{Conclusão}

Para grafos pequenos, a estrutura em matriz é mais recomendada, porém, o uso de memória tem ordem quadrática, ou seja, para grafos muito grandes, o recomendado é o uso de listas dinâmicas. O algoritmo de Dijkstra é indicado em ambos os casos, porém ele é mais rápido utilizando matrizes ao invés de listas, apesar de a primeira ocupar mais memória.

\end{document}